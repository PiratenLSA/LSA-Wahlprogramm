\section{Herabsetzung der 5\%-Hürde bzw. Sperrklausel auf 3\% auf Landesebene}

Die Piratenpartei Sachsen-Anhalt fordert die Herabsetzung der Sperrklausel
in Sachsen-Anhalt bei Landtagswahlen auf 3\%. 

\section{Herabsetzung des aktiven Wahlalters bei Landtagswahlen auf 12 Jahre}

Die Piratenpartei fordert die Senkung des notwendigen Alters zur Wahrnehmung des
aktiven Wahlrechts bei Landtagswahlen auf 12 Jahre und damit eine Anpassung des
§ 42 Abs. 2 der Verfassung des Landes Sachsen-Anhalt. Die erstmalige Ausübung
dieses Wahlrechts erfordert für Unter-16-Jährige die selbständige Eintragung in
eine Wählerliste. Eine Stellvertreterwahl durch Erziehungsberechtigte lehnen wir
ab.

\section{Verbandsklagerecht}

Wir setzen uns für die Einführung eines Verbandsklagerechtes für anerkannte
Tierschutzorganisationen im Sachsen-Anhalt ein. Tiere können als Lebewesen nicht
selbst für ihre Rechte eintreten bzw. diese verteidigen. Daher sind sie auf
Vertreter in Form von Verbänden angewiesen. Obwohl Tier- und Umweltschutz nach
Art. 20a GG denselben Verfassungsrang haben, werden die beiden Staatsziele
ungleich behandelt, wenn es um das Verbandsklagerecht geht. Erfahrungen in
Bremen, wo es die Tierschutzverbandsklage inzwischen gibt, zeigen zudem, dass
die von den Gegnern der Verbandsklage befürchtete Klageflut ausgeblieben ist. Da
auf Bundesebene keine Lösung in Sicht ist, ist die Einführung des
Verbandsklagerechts auf Landesebene geboten.

\section{Mehr und besser ausgestatte Polizeibeamte statt mehr Überwachung}

Statt den Bürgern Sicherheit durch mehr Überwachungsmaßnahmen vorzuspiegeln,
sollten die Gelder dafür in die Beschäftigung von mehr Polizeibeamten investiert
werden. Eine Kamera kann - sofern sie überhaupt von einem Beamten überwacht wird
- keine Hilfe leisten oder herbeirufen. Ein vor Ort patrouillierender Polizist
erhöht die subjektive und die tatsächliche Sicherheit, er kennt die Bewohner
„seines“ Stadtteiles und kann, noch vor der Notwendigkeit von Sanktionen, auf
Mitglieder der Gesellschaft einwirken, die auf die schiefe Bahn zu geraten
drohen.
Allerdings lehnen wir einen Polizeistaat ab. Mehr Personal sollte lediglich in
problematischen Regionen, Orten bzw. Plätzen bereit gestellt werden, oder dort,
wo laufende Ermittlungen durch mangelndes Personal behindert oder gar unmöglich
gemacht werden.
Um der Polizei die Erfüllung ihrer Aufgaben in einem vernünftigen Maße zu
ermöglichen, muss die materielle und personelle Ausstattung verbessert werden.
Die Anschaffung von Ausrüstung wie z. B. Schutzwesten darf nicht dem einzelnen
Polizisten aufgebürdet werden.

\section{Flächendeckendes barrierefreies Notruf- und Informationssystem per
Mobilfunk (SMS-Notruf)}

Die Piratenpartei Sachsen-Anhalt setzt sich für die zeitnahe Einführung eines
flächendeckenden barrierefreien Notruf und Informationssystem per Mobilfunk in
Sachsen-Anhalt ein. Davon profitieren insbesondere gehörlose und schwerhörige
Menschen in Gefahrensituationen. Weiterhin unterstützen wir nach Möglichkeit
alle Bemühungen für eine bundesweite Umsetzung.

\section{Klare Trennung von Kirche und Staat}

Die Piratenpartei setzt sich für eine klare Trennung von Kirche und Staat ein.
Die Piratenpartei ist für Religionsfreiheit und Gleichberechtigung aller
Religionen. Jeder Mensch hat das Recht eine Religion auszuüben, aber jede
Religion ist reine Privatsache jedes Menschen. Die Piratenpartei ist gegen
weitere Alimentierung der Kirchen und Religionsgemeinschaften vom Staat, gegen
das Eintreiben der Kirchensteuer durch den Staat, gegen vom Staat alimentierte
kirchliche Hochschulen, gegen finanzielle Zuschüsse an Kirchen und
Religionsgemeinschaften, gegen Religionsunterricht an staatlichen Schulen, gegen
religiöse Zeichen in Schulen. Im Sinne eines evolutionären Humanismus dürfen
Menschen ohne Religionsbindung nicht gegenüber anderen Menschen benachteiligt
werden und umgekehrt. Die Piratenpartei setzt sich insbesondere auch für die
Ablösung der historisch bedingten Finanztransfers an die Kirchen ein.

\section{Geschlechter- und Familienpolitik}

Die Piratenpartei steht für eine zeitgemäße Geschlechter- und Familienpolitik.
Diese basiert auf dem Prinzip der freien Selbstbestimmung über Angelegenheiten
des persönlichen Lebens. Die Piraten setzen sich dafür ein, dass Politik der
Vielfalt der Lebensstile gerecht wird. Jeder Mensch muß sich frei für den
selbstgewählten Lebensentwurf und für die individuell von ihm gewünschte Form
gleichberechtigten Zusammenlebens entscheiden können. Das Zusammenleben von
Menschen darf nicht auf der Vorteilnahme oder Ausbeutung Einzelner gründen.

\textbf{Freie Selbstbestimmung von geschlechtlicher und sexueller Identität bzw.
Orientierung}

Die Piratenpartei steht für eine Politik, die die freie Selbstbestimmung von
geschlechtlicher und sexueller Identität bzw. Orientierung respektiert und
fördert. Fremdbestimmte Zuordnungen zu einem Geschlecht oder zu
Geschlechterrollen lehnen wir ab. Diskriminierung aufgrund des Geschlechts, der
Geschlechterrolle, der sexuellen Identität oder Orientierung ist Unrecht.
Gesellschaftsstrukturen, die sich aus Geschlechterrollenbildern ergeben, werden
dem Individuum nicht gerecht und sind zu überwinden.
Die Piratenpartei lehnt die Erfassung des Merkmals “Geschlecht” durch staatliche
Behörden ab. Übergangsweise kann die Erfassung seitens des Staates durch eine
von den Individuen selbst vorgenommene Einordnung erfolgen.

\textbf{Freie Selbstbestimmung des Zusammenlebens}

Die Piraten bekennen sich zum Pluralismus des Zusammenlebens. Politik muss der
Vielfalt der Lebensstile gerecht werden und eine wirklich freie Entscheidung für
die individuell gewünschte Form des Zusammenlebens ermöglichen. Eine bloß
historisch gewachsene strukturelle und finanzielle Bevorzugung ausgewählter
Modelle lehnen wir ab.

\textbf{Freie Selbstbestimmung und Familienförderung}

Die Piratenpartei setzt sich für die gleichwertige Anerkennung von
Lebensmodellen ein, in denen Menschen füreinander Verantwortung übernehmen.
Unabhängig vom gewählten Lebensmodell genießen Lebensgemeinschaften, in denen
Kinder aufwachsen oder schwache Menschen versorgt werden, einen besonderen
Schutz. Unsere Familienpolitik ist dadurch bestimmt, dass solche
Lebensgemeinschaften als gleichwertig und als vor dem Gesetz gleich angesehen
werden müssen.
 
\section{Ablehnung von Fracking}

Die Piratenpartei Sachsen-Anhalt lehnt Hydraulic Fracturing, auch Fracking
genannt, als Gasfördermethode ab. Durch diese Methode werden wir und zukünftige
Generationen einem kaum kalkulierbaren Risiko ausgesetzt. Das Einbringen
zahlreicher, zum Teil hochtoxischer Stoffe mit unkontrollierter Ausbreitung ist
abzulehnen. Daher setzen wir uns für ein Verbot von Fracking auf allen
politischen Ebenen ein. Um den Energiebedarf zu decken, setzen wir stattdessen
auf Effizienzverbesserungen, Einsparungen und generative Energien mit modernen
Speichertechniken zum Ausgleich von Fluktuationen bei Energieproduktion und
-verbrauch.

\section{Kulturerhalt und -förderung}

Wie ein demokratisches Gemeinwesen verfasst ist, wird treffend durch die Worte
Friedrich Schillers beschrieben: „Die Kunst ist eine Tochter der Freiheit.“
Durch die Kulturförderung werden nicht nur die Kreativen geschützt, sondern auch
unsere Haltung und Freiheitsrechte. Eine verantwortliche, transparente,
anregende und nachhaltig gestaltende Kulturpolitik kräftigt eine
zukunftsorientierte, vielfältige und humane Gesellschaft. Diese Politik muss die
notwendigen Rahmenbedingungen für eine freie Entfaltung von Kunst und Kultur
schaffen - sie darf diese nicht bewerten oder vereinnahmen.
Die kulturelle Freizügigkeit und Vielfalt sollen durch geförderten Freiraum und
unter Berücksichtigung der Rechte der Anwohner verteidigt werden. Behörden
sollen ihre Ermessensspielräume nutzen, um zugunsten von Kunst- und
Kulturinitiativen zu entscheiden. Das Kulturleben soll sich auch als
Wirtschaftsfaktor und Vernetzungsplattform lebendig weiterentwickeln.
Kulturentwicklungsplanung ist vielschichtig und muss die kulturelle Bildung,
Betätigung und Mitwirkung des Bürgers sowie die Künste und die Kulturwirtschaft
aufeinander abstimmen und die dafür notwendigen Ressourcen und Verfahren
definieren. Die Piratenpartei ist bestrebt, die Förderstruktur von Kunst und
Kultur möglichst stabil zu halten. Bei einzelnen Sparten sollte auch in
Wirtschaftskrisen nicht so stark gekürzt werden, dass ihre jeweilige Existenz
gefährdet ist, denn im Gegensatz zu materiellen Werten kann eine verlorene
kulturelle Infrastruktur nur langsam wieder aufgebaut werden.
Für die PIRATEN steht die Förderung kultureller Vielfalt über der einzelner
Prestigeobjekte. Kleine Kulturprojekte sind meist ehrenamtlich organisiert,
erreichen und beziehen in ihrer Gesamtheit aber deutlich mehr Menschen mit ein.
Der Zugang zu Kultureinrichtungen muss für alle Gesellschaftsschichten offen
gehalten werden, damit diese Institutionen gesellschaftlich verankert sind. Des
Weiteren müssen größtenteils öffentlich finanzierte Einrichtungen auch für die
gesamte Bevölkerung zugänglich sein.

\section{Aufhebung des Veranstaltungsverbots und der Einschränkung der
Versammlungsfreiheit an christlichen Feiertagen durch Abschaffung der §§ 5, 11
FeiertG LSA}

Die Piratenpartei Sachsen-Anhalt strebt die Aufhebung der §§ 5 (Erhöhter Schutz)
und 11 (Einschränkung von Grundrechten) des Gesetzes über die Sonn- und
Feiertage (FeiertG LSA) an. Das Verbot von öffentlichen Veranstaltungen, die
nicht der Würdigung des Feiertages oder der Kunst, Wissenschaft oder
Volksbildung dienen ist abzuschaffen. Weiterhin ist die Einschränkung des
Grundrechts auf Versammlungsfreiheit an religiösen Feiertagen aufzuheben. Die
Trennung von Religion und Staat bzw. die Selbstbestimmung des Individuums ist
höher zu bewerten, als der erhöhte Schutz religiöser Bräuche. Durch Beibehalten
von § 4 (Schutz der Gottesdienste) bleibt der besondere Schutz von
Gottesdiensten jedoch bestehen.

\section{Ungehinderter Zugang zu Verwaltungsdaten}

Die PIRATEN setzen sich für den ungehinderten Zugang zu Protokollen und allen
entscheidungsrelevanten Unterlagen aller Gremien auf kommunaler Ebene ein. Dies
umfasst auch öffentlich-rechtliche Körperschaften wie die kommunalen
Zweckverbände sowie Verträge zwischen staatlichen Stellen und privaten
Unternehmen.
Hierzu soll das Land Sachsen-Anhalt eine geeignete Infrastruktur bzw. Software
bereitstellen, die die unkomplizierte Veröffentlichung der öffentlichen Daten im
Internet ermöglicht. Die Daten müssen auch maschinenlesbar in freien Formaten
unter freien Lizenzen zur Verfügung gestellt werden. Die Weiterverarbeitung,
Aufbereitung und Auswertung durch Dritte ist ausdrücklich erwünscht. Auf einen
barrierefreien Zugang muss besonderer Wert gelegt werden. Die Erstellung einer
freien Software zur Veröffentlichung soll geprüft werden.
Das Informationszugangsgesetz und das Verwaltungskostengesetz des Landes sollen
so erweitert werden, dass die kostenlose Erstellung und Versendung von Kopien
vorgeschrieben werden, wenn die Daten nicht auf die beschriebene Weise
veröffentlicht werden.

\section{Asylpolitik}

\textbf{Bleiberecht}

Es muss eine umfassende Bleiberechtsregelung mit realistischen
Erteilungsvoraussetzungen geben. Das aktive Bemühen von Menschen mit prekärem
Aufenthalt muss durch die Behörden anerkannt werden. Außerdem müssen die Fristen
zur Beantragung von acht Jahren Aufenthalt in Deutschland gesenkt werden um mehr
Menschen neue Möglichkeiten zur selbständigen Lebensunterhaltssicherung zu
ermöglichen.

\textbf{Arbeit}

Um eine gesellschaftliche Teilhabe aller Flüchtlinge und Migranten zu
ermöglichen, sollen alle in Deutschland lebenden Menschen eine Arbeitserlaubnis
erteilt bekommen. Dieses ermöglicht eine selbstständige
Lebensunterhaltssicherung und bereichert den Arbeitsmarkt durch die bisher
ungenutzten Qualifikationen der Menschen ohne Arbeitserlaubnis.

\textbf{Ausbildung / Studium}

Der Zugang zu Ausbildung und Studium für Flüchtlinge und Migranten muss
gleichberechtigt ermöglicht werden um gerade in einer alternden Gesellschaft wie
der Deutschlands die Chancen durch Migration zu nutzen und Perspektiven für alle
zu entwickeln. Außerdem müssen ausländische Schulabschlüsse einfacher anerkannt
werden. Im Schulbereich müssen bundesweit verbindliche Strukturen und
Kapazitäten für Flüchtlingskinder geschaffen werden. Hierzu zählt auch die
Sprachförderung und die Einschulung bis zum 18. Geburtstag.

\textbf{Residenzpflicht}

Diese in Europa einzigartige Regelung muss bundesweit für alle Menschen
abgeschafft werden. Niemand soll in seinem Recht auf freie Bewegungsfreiheit
beschränkt werden. Die Kriminalisierung und Diskriminierung von Flüchtlingen und
Migranten muss aufhören.

\textbf{Medizinische Versorgung}

Der Zugang zu umfassender, unbürokratischer medizinischer Versorgung muss
ermöglicht werden. Das diskriminierende Asylbewerberleitungsgesetz muss
abgeschafft werden und die Menschen müssen Mitglied einer gesetzlichen
Krankenkasse werden. Ein erfolgreiches Modell findet sich in Bremen.

\textbf{Unterbringung}

Unbürokratische Zusicherungen der Mietkostenübernahme durch das Sozialamt in
Verbindung mit einer generell-en Übernahme der Mietkaution als zinslosem Kredit.

\textbf{Ausländerbehörde}

Die soziale, fachliche und sprachliche Kompetenzen der Sachbearbeitern muss
ausgebaut werden. Die ABH soll nicht nur restriktiv agieren, sondern die
Menschen fördern und Teilhabe ermöglichen.

\section{Fachärztemangel}

Der Landesverband Sachsen-Anhalt setzt sich dafür ein, dem Landärztemangel
gegenzusteuern. Seit Jahren ist die Zahl der praktizierenden Ärzte auf dem Land
rückläufig. Das führt zu einer gravierenden Unterversorgung der gesundheitlichen
Betreuung in ländlichen Regionen. Um diesem Mangel an Ärzten entgegenzusteuern,
bedarf es umfangreicher struktureller Maßnahmen. Dazu gehört:
\begin{itemize}
\item der Beruf Praktischer Arzt muss wieder eingeführt werden
\item der bürokratische Aufwand für Hausärzte muss erheblich erleichtert werden
\item die angehenden Landärzte erfahren finanzielle und materielle Unterstützung
bei der Einrichtung einer Praxis und erhalten ein permanentes Grundgehalt,
welches gleich oder höher des regionalen Durchschnittsgehaltes eines Facharztes
ist.
\item der ärztliche Leistungskatalog muss zugunsten der ärztlichen
Grundversorgung überarbeitet werden
\end{itemize}
Zur Umsetzung dieser Maßnahmen sind umgehend Kommissionen einzusetzen,
bestehend aus Fachleuten vom KVSA, dem Hausärzteverband Sachsen-Anhalt e.V.,
sowie dem Hartmannbund, die die anstehenden Probleme benennen, damit die Politik
die zeitnahe Behebung auf den Weg bringen kann.

Als weitere Maßnahmen wird empfohlen:
\begin{itemize}
\item die Einführung eines nichtrückzahlbaren Zusatzstipendiums. Dieses geht
einher mit der Verpflichtung, für die Dauer der Zahlung anschließend auf dem
Land zu arbeiten.
\item die Neuordnung des Bereitschaftsdienstes sowie
\item die Unterstützung von Familienmitgliedern bei der Erwerbstätigkeit 
\end{itemize}

\section{Bildungspolitik ist Bundespolitik}

Die Piratenpartei Sachsen Anhalt setzt sich dafür ein, dass das Land Sachsen
Anhalt in der Kultusministerkonferenz für eine einheitliche Bildungspolitik auf
Bundesebene eintritt, und das Kooperationsverbot aufgehoben wird.

\section{Ablehnung Leistungsorientierte Mittelvergabe an Hochschulen}

In Sachsen-Anhalt werden die Hochschulen derzeit in Wettbewerben untereinander
gemessen. Diese Wettbewerbe orientieren sich an verschiedenen Zielvereinbarungen
Diese Zielvereinbarungen enthalten Punkte zur Messung der „Leistung“ einer
Hochschule wie z.B.:
\begin{itemize}
\item Abbrecher- \& Absolventenquoten
\item Kooperation Wirtschaft / Wissenschaft
\item Profilbildung
\item Gleichberechtigung (Anzahl der Absolventinnen, Professorinnen und
Mitarbeiterinnen)
\end{itemize}
Damit die Hochschulen „motiviert“ werden, an diesem Wettbewerb teilzunehmen,
wird die Teil-Auszahlung der jeweiligen Hochschulbudgets von bestimmten
"Leistungenäbhängig gemacht. Der Anteil des betreffenden Budgets steigt aktuell
an, von 5\% in 2011, über 2012 10\% bis ins Jahr 2013 auf 15\% des Gesamtbudgets
der jeweiligen Hochschule (abzüglich der Mittel für Investitionen) Wir sind der
Meinung, dass diese Art der Leistungsorientierten Mittelvergabe die
Universitäten in ihrer Aufgabe als unabhängige Institution für die Bildung und
Forschung stark einschränken. Aus solch ein Wettbewerb profitieren vor allem die
ingenieurwissenschaftlichen Fachrichtungen.
Die Hochschulen in Sachsen-Anhalt sollen 100\% ihrer Finanzierung erhalten, ohne
in einen Wettbewerb treten zu müssen. Die Piratenpartei Sachsen-Anhalt lehnt die
Leistungsorientierte Mittelvergabe als eine Finanzierungsart für die Hochschulen
im Land grundsätzlich ab.

\section{Neofaschismus}

Die Piratenpartei Sachsen-Anhalt strebt eine vielfältige Gesellschaft an, in der
sich alle Menschen im Rahmen eines friedlichen Zusammenlebens frei nach ihren
eigenen Bedürfnissen entfalten können. Daher beziehen wir explizit Stellung
gegen menschenverachtende Weltanschauungen in organisierter, nicht organisierter
als auch in alltäglicher Form.
